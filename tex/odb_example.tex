\pgfplotstabletypeset[
   empty header,
    begin table=\begin{longtable},
    %every head row/.style={output empty row},
    every nth row={1}{before row=\hline},
    every first row/.append style={
        before row={%
            % Initial caption
            \caption{Simplified example of records in ODB type data model, with observations from reports 1 and 2 spanning multiple records. For simplicity, the z coordinate has been omitted but profile data would be represented with each layer / height as a separate record}
            \label{tab:DataTableODB}\\
            % Initial column headers
            \hline\hline
            \multicolumn{5} {c} { \textbf{header information}} & 
            \multicolumn{3} {c} { \textbf{observation information}} 
            \\
            \hline
            \multicolumn{1} { > {\centering}V{0.3 in}} { \textbf{record id}} & 
            \multicolumn{1} { > {\centering}V{0.3 in}} { \textbf{report id}} & 
            \multicolumn{1} { > {\centering}V{0.3 in}} { \textbf{obs id}} & 
            \multicolumn{1} { > {\centering}V{1 in}} { \textbf{date}} & 
            \multicolumn{1} { > {\centering}V{1 in}} { \textbf{location}} & 
            \multicolumn{1} { > {\centering}V{0.75 in}} { \textbf{parameter}} & 
            \multicolumn{1} { > {\centering} V{0.75 in} } {\textbf{value}} &
            \multicolumn{1} { > {\centering} V{0.75 in} } {\textbf{units}} 
            \\ \hline\hline \endfirsthead
            \multicolumn{6}{c}{Table \thetable\ adjustment (cont.)} \\
            % column headers on additional pages
            \hline\hline 
            \multicolumn{5} { > {\centering}V{2.9 in}} { \textbf{header information}} & 
            \multicolumn{3} { > {\centering}V{2.25 in}} { \textbf{observation information}} 
            \\
            \hline
            \multicolumn{1} { > {\centering}V{0.3 in}} { \textbf{record id}} & 
            \multicolumn{1} { > {\centering}V{0.3 in}} { \textbf{report id}} & 
            \multicolumn{1} { > {\centering}V{0.3 in}} { \textbf{obs id}} & 
            \multicolumn{1} { > {\centering}V{1 in}} { \textbf{date}} & 
            \multicolumn{1} { > {\centering}V{1 in}} { \textbf{location}} & 
            \multicolumn{1} { > {\centering}V{0.75 in}} { \textbf{parameter}} & 
            \multicolumn{1} { > {\centering} V{0.75 in} } {\textbf{value}} &
            \multicolumn{1} { > {\centering} V{0.75 in} } {\textbf{units}} 
            \\ \hline\hline \endhead
            % Footer on 1st to penultimate pages
            \multicolumn{6}{r}{{Continued on next page}} \\
            \endfoot
            % Footer on last page of table
            \hline
            \multicolumn{6}{r}{{End of table}} \\ 
            \endlastfoot
        }
    },
    %
    end table=\end{longtable},
    col sep = &,
    row sep = \\ , 
    columns/record id/.style={string type, column type = V{0.3 in}},
    columns/report id/.style={string type, column type = V{0.3 in}},
    columns/observation id/.style={string type, column type = V{0.3 in}},
    columns/date/.style={string type, column type = V{1 in}},
    columns/location/.style={string type, column type = V{1 in}}, 
    columns/parameter/.style={string type, column type = V{1 in}},
    columns/value/.style={string type, column type = V{0.3 in}},
    columns/units/.style={string type, column type = V{0.3 in}}
]{
record id& report id& observation id& date & location & parameter & value & units \\
1 & 1 & 1 & 2012-01-01 12:00+0.0 & POINT(-40 40) & air temperature & 300.0 & K \\
2 & 1 & 2 & 2012-01-01 12:00+0.0 & POINT(-40 40) & sea level pressure & 1013.0 & hPa \\
3 & 2 & 3 & 2012-01-01 18:00+0.0 & POINT(-40.1 40.2) & air temperature & 300.3 & K \\
4 & 2 & 4 & 2012-01-01 18:00+0.0 & POINT(-40.1 40.2) & sea level pressure & 1013.2 & hPa \\
}

\documentclass[a4paper]{article}
\usepackage{pgfplotstable}
\usepackage{longtable}
\usepackage{varwidth}
\usepackage{array}
\usepackage{lscape}
\usepackage[top=2 cm,bottom=2cm]{geometry}



\title {Copernicus Climate Change Service - 311a Lot 2 \\ Defining a Common Data Model}
\author {David I. Berry \\ National Oceanography Centre, UK}
\date {28th June 2017}


\pgfplotstableset{
empty header/.style={
           typeset cell/.append code={%
               \ifnum\pgfplotstablerow=-1 %
                   \pgfkeyssetvalue{/pgfplots/table/@cell content}{}%
               \fi
           }
       }
   }

\begin{document}


\maketitle

\vskip 0.5in
\hrule
Summary ...
\vskip 2in
\hrule

\newpage
\tableofcontents
\newpage
\listoftables
\newpage


\section {Introduction}

The Copernicus Climate Change Service (C3S), through its "Collection and Processing of In Situ Observations (C3S 311a)" tender, seeks to harmonise both data formats and metadata (discovery and observational) conventions. The first step of this process, as noted within the invitation to tender, is the development or adoption of a common data model  for the data and metadata. Within this document, when complete, we will describe the common data model developed within Lot(s) 1 - 4 of the C3S 311a tender in consultation with ECMWF. The themes for the Lots 1 - 4 are:\\
\begin{itemize}
\item Lot 1 - Coordination of data rescue activities
\item Lot 2 - Access to observations from global climate data archives
\item Lot 3 - Access to observations from baseline and reference networks
\item Lot 4 - Climate monitoring products for Europe based on in situ observations.
\end{itemize}
Lot 1:\\

Within Lot 2, observations and metadata from land stations and marine platforms will be harmonised into a common data model and a web based service developed to serve the data through the C3S Climate Data Store (CDS). The observations include instantaneous / point observations, such as those from SYNOP weather reports, as well as daily and monthly summaries (CLIMAT DAILY and CLIMAT). A single report may contain observations of multiple parameters, e.g. air temperature, humidity, wind speed etc. The stations range from stationary land stations to mobile merchant ships, drifting buoys and other marine platforms.\\

Lot 3: Atmospheric profile data from GRUAN. Multiple observations of the same parameter(s) at different heights in a single report ? \\

Lot 4: Users of data extracted from CDS?\\

Section 2 of this report provides background information on joint activities between Lots 2 and 3 so far, the ECMWF Observations DataBase (ODB) data model and relevant WMO data models. Section 3 gives an overview of the preferred data model from Lot 2 and proposes a list of elements for the observations table. Auxiliary tables are also proposed in Section 3 but left empty for future discussion once the principles of the type of data model have been agreed across lots. Section 4 proposes a governance mechanism for the common data model across lots and next steps required.\\

\section {Observations table}

Preamble text ... \\

\begin{landscape}
\pgfplotstabletypeset[
    empty header,
    begin table=\begin{longtable},
    %every head row/.style={output empty row},
    every nth row={1}{before row=\hline},
    every first row/.append style={
        before row={%
            % Initial caption
            \caption{observations\_table}
            \label{tab:DataTableObservationstable}\\
            % Initial column headers
            \hline\hline \multicolumn{1} { > {\centering}V{1.675000 in}} { \textbf{element\_number}} & 
\multicolumn{1} { > {\centering}V{1.675000 in}} { \textbf{element\_name}} & 
\multicolumn{1} { > {\centering}V{1.675000 in}} { \textbf{kind}} & 
\multicolumn{1} { > {\centering}V{1.675000 in}} { \textbf{external\_table}} & 
 \multicolumn{1} { > {\centering} V{3.000000 in} } {\textbf{description}} \\ \hline\hline \endfirsthead
            \multicolumn{5}{c}{Table \thetable\ observations\_table (cont.)} \\
            % column headers on additional pages
            \hline\hline \multicolumn{1} { > {\centering}V{1.675000 in} } { \textbf{element\_number}} & 
\multicolumn{1} { > {\centering}V{1.675000 in} } { \textbf{element\_name}} & 
\multicolumn{1} { > {\centering}V{1.675000 in} } { \textbf{kind}} & 
\multicolumn{1} { > {\centering}V{1.675000 in} } { \textbf{external\_table}} & 
 \multicolumn{1} { > {\centering} V{3.000000 in} } {\textbf{description}} \\ \hline\hline \endhead
            % Footer on 1st to penultimate pages
            \multicolumn{5}{r}{{Continued on next page}} \\
            \endfoot
            % Footer on last page of table
            \hline
            \multicolumn{5}{r}{{End of table}} \\ 
            \endlastfoot
        }
    },
    %
    end table=\end{longtable},
    col sep=tab,
    string type,
    columns/element_number/.style={
            string type, 
            column type= V{1.675000 in}, 
            string replace*={_}{\_}
        },
    columns/element_name/.style={
            string type, 
            column type= V{1.675000 in}, 
            string replace*={_}{\_}
        },
    columns/kind/.style={
            string type, 
            column type= V{1.675000 in}, 
            string replace*={_}{\_}
        },
    columns/external_table/.style={
            string type, 
            column type= V{1.675000 in}, 
            string replace*={_}{\_}
        },
    columns/description/.style={
            string type, 
            string replace*={_}{\_},
            column type = V{3.000000 in}
        }
    ]{/Users/dyb/GitHub/C3S_311a_CDM/tables/observations_table.csv}
\end{landscape}


\section {Station configuration table}

Entity-attribute value based table for station configuration (and others).

\section {Source configuration table}
\section {Profile configuration table}
\section {Sensor configuration table}
\section {Code tables}

\input {region}
\pgfplotstabletypeset[
    empty header,
    begin table=\begin{longtable},
    %every head row/.style={output empty row},
    every nth row={1}{before row=\hline},
    every first row/.append style={
        before row={%
            % Initial caption
            \caption{sub\_region (NA)}
            \label{tab:DataTableSubregion}\\
            % Initial column headers
            \hline\hline             \multicolumn{1} { V{0.933333 in}} { \textbf{\seqsplit{sub\_region}}} & 
            \multicolumn{1} { V{0.933333 in}} { \textbf{\seqsplit{type}}} & 
            \multicolumn{1} { V{0.933333 in}} { \textbf{\seqsplit{code}}} & 
            \multicolumn{1} { V{4.000000 in} } {\textbf{\seqsplit{name}}} \\ \hline\hline \endfirsthead
            \multicolumn{4}{c}{Table \thetable\ sub\_region (cont.)} \\
            % column headers on additional pages
            \hline\hline             \multicolumn{1} {V{0.933333 in} } { \textbf{\seqsplit{sub\_region}}} & 
            \multicolumn{1} {V{0.933333 in} } { \textbf{\seqsplit{type}}} & 
            \multicolumn{1} {V{0.933333 in} } { \textbf{\seqsplit{code}}} & 
            \multicolumn{1} { V{4.000000 in} } {\textbf{\seqsplit{name}}} \\ \hline\hline \endhead
            % Footer on 1st to penultimate pages
            \multicolumn{4}{r}{{Continued on next page}} \\
            \endfoot
            % Footer on last page of table
            \hline
            \multicolumn{4}{r}{{End of table}} \\ 
            \endlastfoot
        }
    },
    %
    end table=\end{longtable},
    col sep=tab,
    string type,
    columns/sub_region/.style={
            string type, 
            column type= V{0.933333 in}, 
            string replace*={_}{\_}
        },
    columns/type/.style={
            string type, 
            column type= V{0.933333 in}, 
            string replace*={_}{\_}
        },
    columns/code/.style={
            string type, 
            column type= V{0.933333 in}, 
            string replace*={_}{\_}
        },
    columns/name/.style={
            string type, 
            string replace*={_}{\_},
            column type = V{4.000000 in}
        }
    ]{/Users/dyb/GitHub/C3S_311a_CDM/tables/sub_region.csv}

\input {application_area}
\begin{landscape}
\pgfplotstabletypeset[
    empty header,
    begin table=\begin{longtable},
    %every head row/.style={output empty row},
    every nth row={1}{before row=\hline},
    every first row/.append style={
        before row={%
            % Initial caption
            \caption{observing\_programme (WIGOS Code Table 2-02)}
            \label{tab:DataTableObservingprogramme}\\
            % Initial column headers
            \hline\hline             \multicolumn{1} { V{1.475000 in}} { \textbf{\seqsplit{index}}} & 
            \multicolumn{1} { V{1.475000 in}} { \textbf{\seqsplit{observing\_programme}}} & 
            \multicolumn{1} { V{1.475000 in}} { \textbf{\seqsplit{abbreviation}}} & 
            \multicolumn{1} { V{1.475000 in}} { \textbf{\seqsplit{description}}} & 
            \multicolumn{1} { V{3.000000 in} } {\textbf{\seqsplit{sponsor}}} \\ \hline\hline \endfirsthead
            \multicolumn{5}{c}{Table \thetable\ observing\_programme (cont.)} \\
            % column headers on additional pages
            \hline\hline             \multicolumn{1} {V{1.475000 in} } { \textbf{\seqsplit{index}}} & 
            \multicolumn{1} {V{1.475000 in} } { \textbf{\seqsplit{observing\_programme}}} & 
            \multicolumn{1} {V{1.475000 in} } { \textbf{\seqsplit{abbreviation}}} & 
            \multicolumn{1} {V{1.475000 in} } { \textbf{\seqsplit{description}}} & 
            \multicolumn{1} { V{3.000000 in} } {\textbf{\seqsplit{sponsor}}} \\ \hline\hline \endhead
            % Footer on 1st to penultimate pages
            \multicolumn{5}{r}{{Continued on next page}} \\
            \endfoot
            % Footer on last page of table
            \hline
            \multicolumn{5}{r}{{End of table}} \\ 
            \endlastfoot
        }
    },
    %
    end table=\end{longtable},
    col sep=tab,
    string type,
    columns/index/.style={
            string type, 
            column type= V{1.475000 in}, 
            string replace*={_}{\_}
        },
    columns/observing_programme/.style={
            string type, 
            column type= V{1.475000 in}, 
            string replace*={_}{\_}
        },
    columns/abbreviation/.style={
            string type, 
            column type= V{1.475000 in}, 
            string replace*={_}{\_}
        },
    columns/description/.style={
            string type, 
            column type= V{1.475000 in}, 
            string replace*={_}{\_}
        },
    columns/sponsor/.style={
            string type, 
            string replace*={_}{\_},
            column type = V{3.000000 in}
        }
    ]{/Users/dyb/GitHub/common_data_model/tables/observing_programme.csv}
\end{landscape}

\input {report_type}
\pgfplotstabletypeset[
    empty header,
    begin table=\begin{longtable},
    %every head row/.style={output empty row},
    every nth row={1}{before row=\hline},
    every first row/.append style={
        before row={%
            % Initial caption
            \caption{station\_type}
            \label{tab:DataTableStationtype}\\
            % Initial column headers
            \hline\hline             \multicolumn{1} { V{1.900000 in}} { \textbf{\seqsplit{index}}} & 
            \multicolumn{1} { V{1.900000 in}} { \textbf{\seqsplit{station\_type}}} & 
            \multicolumn{1} { V{3.000000 in} } {\textbf{\seqsplit{description}}} \\ \hline\hline \endfirsthead
            \multicolumn{3}{c}{Table \thetable\ station\_type (cont.)} \\
            % column headers on additional pages
            \hline\hline             \multicolumn{1} {V{1.900000 in} } { \textbf{\seqsplit{index}}} & 
            \multicolumn{1} {V{1.900000 in} } { \textbf{\seqsplit{station\_type}}} & 
            \multicolumn{1} { V{3.000000 in} } {\textbf{\seqsplit{description}}} \\ \hline\hline \endhead
            % Footer on 1st to penultimate pages
            \multicolumn{3}{r}{{Continued on next page}} \\
            \endfoot
            % Footer on last page of table
            \hline
            \multicolumn{3}{r}{{End of table}} \\ 
            \endlastfoot
        }
    },
    %
    end table=\end{longtable},
    col sep=tab,
    string type,
    columns/index/.style={
            string type, 
            column type= V{1.900000 in}, 
            string replace*={_}{\_}
        },
    columns/station_type/.style={
            string type, 
            column type= V{1.900000 in}, 
            string replace*={_}{\_}
        },
    columns/description/.style={
            string type, 
            string replace*={_}{\_},
            column type = V{3.000000 in}
        }
    ]{/Users/dyb/GitHub/common_data_model/tables/station_type.csv}

\pgfplotstabletypeset[
    empty header,
    begin table=\begin{longtable},
    %every head row/.style={output empty row},
    every nth row={1}{before row=\hline},
    every first row/.append style={
        before row={%
            % Initial caption
            \caption{platform\_type (IMMA (ICOADS) and BUFR 0 03 001)}
            \label{tab:DataTablePlatformtype}\\
            % Initial column headers
            \hline\hline             \multicolumn{1} { V{3.800000 in}} { \textbf{\seqsplit{platform\_type}}} & 
            \multicolumn{1} { V{3.000000 in} } {\textbf{\seqsplit{description}}} \\ \hline\hline \endfirsthead
            \multicolumn{2}{c}{Table \thetable\ platform\_type (cont.)} \\
            % column headers on additional pages
            \hline\hline             \multicolumn{1} {V{3.800000 in} } { \textbf{\seqsplit{platform\_type}}} & 
            \multicolumn{1} { V{3.000000 in} } {\textbf{\seqsplit{description}}} \\ \hline\hline \endhead
            % Footer on 1st to penultimate pages
            \multicolumn{2}{r}{{Continued on next page}} \\
            \endfoot
            % Footer on last page of table
            \hline
            \multicolumn{2}{r}{{End of table}} \\ 
            \endlastfoot
        }
    },
    %
    end table=\end{longtable},
    col sep=tab,
    string type,
    columns/platform_type/.style={
            string type, 
            column type= V{3.800000 in}, 
            string replace*={_}{\_}
        },
    columns/description/.style={
            string type, 
            string replace*={_}{\_},
            column type = V{3.000000 in}
        }
    ]{/Users/dyb/GitHub/C3S_311a_CDM/tables/platform_type.csv}

\begin{landscape}
\pgfplotstabletypeset[
    empty header,
    begin table=\begin{longtable},
    %every head row/.style={output empty row},
    every nth row={1}{before row=\hline},
    every first row/.append style={
        before row={%
            % Initial caption
            \caption{platform\_sub\_type}
            \label{tab:DataTablePlatformsubtype}\\
            % Initial column headers
            \hline\hline             \multicolumn{1} { V{1.475000 in}} { \textbf{\seqsplit{index}}} & 
            \multicolumn{1} { V{1.475000 in}} { \textbf{\seqsplit{platform\_sub\_type}}} & 
            \multicolumn{1} { V{1.475000 in}} { \textbf{\seqsplit{platform\_type}}} & 
            \multicolumn{1} { V{1.475000 in}} { \textbf{\seqsplit{abbreviation}}} & 
            \multicolumn{1} { V{3.000000 in} } {\textbf{\seqsplit{description}}} \\ \hline\hline \endfirsthead
            \multicolumn{5}{c}{Table \thetable\ platform\_sub\_type (cont.)} \\
            % column headers on additional pages
            \hline\hline             \multicolumn{1} {V{1.475000 in} } { \textbf{\seqsplit{index}}} & 
            \multicolumn{1} {V{1.475000 in} } { \textbf{\seqsplit{platform\_sub\_type}}} & 
            \multicolumn{1} {V{1.475000 in} } { \textbf{\seqsplit{platform\_type}}} & 
            \multicolumn{1} {V{1.475000 in} } { \textbf{\seqsplit{abbreviation}}} & 
            \multicolumn{1} { V{3.000000 in} } {\textbf{\seqsplit{description}}} \\ \hline\hline \endhead
            % Footer on 1st to penultimate pages
            \multicolumn{5}{r}{{Continued on next page}} \\
            \endfoot
            % Footer on last page of table
            \hline
            \multicolumn{5}{r}{{End of table}} \\ 
            \endlastfoot
        }
    },
    %
    end table=\end{longtable},
    col sep=tab,
    string type,
    columns/index/.style={
            string type, 
            column type= V{1.475000 in}, 
            string replace*={_}{\_}
        },
    columns/platform_sub_type/.style={
            string type, 
            column type= V{1.475000 in}, 
            string replace*={_}{\_}
        },
    columns/platform_type/.style={
            string type, 
            column type= V{1.475000 in}, 
            string replace*={_}{\_}
        },
    columns/abbreviation/.style={
            string type, 
            column type= V{1.475000 in}, 
            string replace*={_}{\_}
        },
    columns/description/.style={
            string type, 
            string replace*={_}{\_},
            column type = V{3.000000 in}
        }
    ]{/Users/dyb/GitHub/common_data_model/tables/platform_sub_type.csv}
\end{landscape}

\input {id_scheme}
\pgfplotstabletypeset[
    empty header,
    begin table=\begin{longtable},
    %every head row/.style={output empty row},
    every nth row={1}{before row=\hline},
    every first row/.append style={
        before row={%
            % Initial caption
            \caption{location\_method}
            \label{tab:DataTableLocationmethod}\\
            % Initial column headers
            \hline\hline             \multicolumn{1} { V{1.900000 in}} { \textbf{\seqsplit{index}}} & 
            \multicolumn{1} { V{1.900000 in}} { \textbf{\seqsplit{location\_method}}} & 
            \multicolumn{1} { V{3.000000 in} } {\textbf{\seqsplit{description}}} \\ \hline\hline \endfirsthead
            \multicolumn{3}{c}{Table \thetable\ location\_method (cont.)} \\
            % column headers on additional pages
            \hline\hline             \multicolumn{1} {V{1.900000 in} } { \textbf{\seqsplit{index}}} & 
            \multicolumn{1} {V{1.900000 in} } { \textbf{\seqsplit{location\_method}}} & 
            \multicolumn{1} { V{3.000000 in} } {\textbf{\seqsplit{description}}} \\ \hline\hline \endhead
            % Footer on 1st to penultimate pages
            \multicolumn{3}{r}{{Continued on next page}} \\
            \endfoot
            % Footer on last page of table
            \hline
            \multicolumn{3}{r}{{End of table}} \\ 
            \endlastfoot
        }
    },
    %
    end table=\end{longtable},
    col sep=tab,
    string type,
    columns/index/.style={
            string type, 
            column type= V{1.900000 in}, 
            string replace*={_}{\_}
        },
    columns/location_method/.style={
            string type, 
            column type= V{1.900000 in}, 
            string replace*={_}{\_}
        },
    columns/description/.style={
            string type, 
            string replace*={_}{\_},
            column type = V{3.000000 in}
        }
    ]{/Users/dyb/GitHub/C3S_311a_CDM/tables/location_method.csv}

%\include {location_quality}
\input {crs}
%\include {surface_type}
% \include {surface_type_scheme}
% \incude {topography}
% \include {station_configuration}
\input {sea_level_datum}
\input {meaning_of_time_stamp}
\input {time_quality}
\input {time_reference}
%\pgfplotstabletypeset[
    empty header,
    begin table=\begin{longtable},
    %every head row/.style={output empty row},
    every nth row={1}{before row=\hline},
    every first row/.append style={
        before row={%
            % Initial caption
            \caption{Profile configuration}
            \label{tab:DataTable}\\
            % Initial column headers

\input {events_at_station}
\pgfplotstabletypeset[
    empty header,
    begin table=\begin{longtable},
    %every head row/.style={output empty row},
    every nth row={1}{before row=\hline},
    every first row/.append style={
        before row={%
            % Initial caption
            \caption{Quality flag}
            \label{tab:DataTable}\\
            % Initial column headers
            \hline\hline \multicolumn{1} { > {\centering}V{3.300000 in}} { \textbf{Value}} & 
 \multicolumn{1} { > {\centering} V{3.000000 in} } {\textbf{Description}} \\ \hline\hline \endfirsthead
            \multicolumn{2}{c}{Table \thetable\ Quality flag (cont.)} \\
            % column headers on additional pages
            \hline\hline \multicolumn{1} { > {\centering}V{3.300000 in} } { \textbf{Value}} & 
 \multicolumn{1} { > {\centering} V{3.000000 in} } {\textbf{Description}} \\ \hline\hline \endhead
            % Footer on 1st to penultimate pages
            \multicolumn{2}{r}{{Continued on next page}} \\
            \endfoot
            % Footer on last page of table
            \hline
            \multicolumn{2}{r}{{End of table}} \\ 
            \endlastfoot
        }
    },
    %
    end table=\end{longtable},
    col sep=tab,
    string type,
    columns/Value/.style={
            string type, 
            column type= V{3.300000 in}, 
            string replace*={_}{}
        },
    columns/Description/.style={
            string type, 
            string replace*={_}{},
            column type = V{3.000000 in}
        }
    ]{/Users/dyb/GitHub/C3S_311a_CDM/tables/quality_flag.csv}

\input {duplicate_status}
\input {update_frequency}
%\pgfplotstabletypeset[
    empty header,
    begin table=\begin{longtable},
    %every head row/.style={output empty row},
    every nth row={1}{before row=\hline},
    every first row/.append style={
        before row={%
            % Initial caption
            \caption{report\_history (NA)}
            \label{tab:DataTableReporthistory}\\
            % Initial column headers
            \hline\hline             \multicolumn{1} { V{1.900000 in}} { \textbf{\seqsplit{element\_name}}} & 
            \multicolumn{1} { V{1.900000 in}} { \textbf{\seqsplit{type}}} & 
            \multicolumn{1} { V{3.000000 in} } {\textbf{\seqsplit{Description}}} \\ \hline\hline \endfirsthead
            \multicolumn{3}{c}{Table \thetable\ report\_history (cont.)} \\
            % column headers on additional pages
            \hline\hline             \multicolumn{1} {V{1.900000 in} } { \textbf{\seqsplit{element\_name}}} & 
            \multicolumn{1} {V{1.900000 in} } { \textbf{\seqsplit{type}}} & 
            \multicolumn{1} { V{3.000000 in} } {\textbf{\seqsplit{Description}}} \\ \hline\hline \endhead
            % Footer on 1st to penultimate pages
            \multicolumn{3}{r}{{Continued on next page}} \\
            \endfoot
            % Footer on last page of table
            \hline
            \multicolumn{3}{r}{{End of table}} \\ 
            \endlastfoot
        }
    },
    %
    end table=\end{longtable},
    col sep=tab,
    string type,
    columns/element_name/.style={
            string type, 
            column type= n{1.900000 in}, 
            string replace*={_}{\_}
        },
    columns/type/.style={
            string type, 
            column type= V{1.900000 in}, 
            string replace*={_}{\_}
        },
    columns/Description/.style={
            string type, 
            string replace*={_}{\_},
            column type = V{3.000000 in}
        }
    ]{/Users/dyb/GitHub/C3S_311a_CDM/tables/report_history.csv}

%\include {report_processing_level}
%\include {report_processing_code}
%\include {source_configuration}
\pgfplotstabletypeset[
    empty header,
    begin table=\begin{longtable},
    %every head row/.style={output empty row},
    every nth row={1}{before row=\hline},
    every first row/.append style={
        before row={%
            % Initial caption
            \caption{data\_policy\_licence (WIGOS 9-02)}
            \label{tab:DataTableDatapolicylicence}\\
            % Initial column headers
            \hline\hline             \multicolumn{1} { V{1.900000 in}} { \textbf{\seqsplit{data\_policy\_licence}}} & 
            \multicolumn{1} { V{1.900000 in}} { \textbf{\seqsplit{name}}} & 
            \multicolumn{1} { V{3.000000 in} } {\textbf{\seqsplit{description}}} \\ \hline\hline \endfirsthead
            \multicolumn{3}{c}{Table \thetable\ data\_policy\_licence (cont.)} \\
            % column headers on additional pages
            \hline\hline             \multicolumn{1} {V{1.900000 in} } { \textbf{\seqsplit{data\_policy\_licence}}} & 
            \multicolumn{1} {V{1.900000 in} } { \textbf{\seqsplit{name}}} & 
            \multicolumn{1} { V{3.000000 in} } {\textbf{\seqsplit{description}}} \\ \hline\hline \endhead
            % Footer on 1st to penultimate pages
            \multicolumn{3}{r}{{Continued on next page}} \\
            \endfoot
            % Footer on last page of table
            \hline
            \multicolumn{3}{r}{{End of table}} \\ 
            \endlastfoot
        }
    },
    %
    end table=\end{longtable},
    col sep=tab,
    string type,
    columns/data_policy_licence/.style={
            string type, 
            column type= V{1.900000 in}, 
            string replace*={_}{\_}
        },
    columns/name/.style={
            string type, 
            column type= n{1.900000 in}, 
            string replace*={_}{\_}
        },
    columns/description/.style={
            string type, 
            string replace*={_}{\_},
            column type = V{3.000000 in}
        }
    ]{/Users/dyb/GitHub/C3S_311a_CDM/tables/data_policy_licence.csv}

\begin{landscape}
\pgfplotstabletypeset[
    empty header,
    begin table=\begin{longtable},
    %every head row/.style={output empty row},
    every nth row={1}{before row=\hline},
    every first row/.append style={
        before row={%
            % Initial caption
            \caption{observed\_variable}
            \label{tab:DataTableObservedvariable}\\
            % Initial column headers
            \hline\hline             \multicolumn{1} { V{0.737500 in}} { \textbf{\seqsplit{index}}} & 
            \multicolumn{1} { V{0.737500 in}} { \textbf{\seqsplit{observed\_variable}}} & 
            \multicolumn{1} { V{0.737500 in}} { \textbf{\seqsplit{parameter\_group}}} & 
            \multicolumn{1} { V{0.737500 in}} { \textbf{\seqsplit{domain}}} & 
            \multicolumn{1} { V{0.737500 in}} { \textbf{\seqsplit{sub\_domain}}} & 
            \multicolumn{1} { V{0.737500 in}} { \textbf{\seqsplit{abbreviation}}} & 
            \multicolumn{1} { V{0.737500 in}} { \textbf{\seqsplit{name}}} & 
            \multicolumn{1} { V{0.737500 in}} { \textbf{\seqsplit{units}}} & 
            \multicolumn{1} { V{2.000000 in} } {\textbf{\seqsplit{description}}} \\ \hline\hline \endfirsthead
            \multicolumn{9}{c}{Table \thetable\ observed\_variable (cont.)} \\
            % column headers on additional pages
            \hline\hline             \multicolumn{1} {V{0.737500 in} } { \textbf{\seqsplit{index}}} & 
            \multicolumn{1} {V{0.737500 in} } { \textbf{\seqsplit{observed\_variable}}} & 
            \multicolumn{1} {V{0.737500 in} } { \textbf{\seqsplit{parameter\_group}}} & 
            \multicolumn{1} {V{0.737500 in} } { \textbf{\seqsplit{domain}}} & 
            \multicolumn{1} {V{0.737500 in} } { \textbf{\seqsplit{sub\_domain}}} & 
            \multicolumn{1} {V{0.737500 in} } { \textbf{\seqsplit{abbreviation}}} & 
            \multicolumn{1} {V{0.737500 in} } { \textbf{\seqsplit{name}}} & 
            \multicolumn{1} {V{0.737500 in} } { \textbf{\seqsplit{units}}} & 
            \multicolumn{1} { V{2.000000 in} } {\textbf{\seqsplit{description}}} \\ \hline\hline \endhead
            % Footer on 1st to penultimate pages
            \multicolumn{9}{r}{{Continued on next page}} \\
            \endfoot
            % Footer on last page of table
            \hline
            \multicolumn{9}{r}{{End of table}} \\ 
            \endlastfoot
        }
    },
    %
    end table=\end{longtable},
    col sep=tab,
    string type,
    columns/index/.style={
            string type, 
            column type= V{0.737500 in}, 
            string replace*={_}{\_}
        },
    columns/observed_variable/.style={
            string type, 
            column type= V{0.737500 in}, 
            string replace*={_}{\_}
        },
    columns/parameter_group/.style={
            string type, 
            column type= V{0.737500 in}, 
            string replace*={_}{\_}
        },
    columns/domain/.style={
            string type, 
            column type= V{0.737500 in}, 
            string replace*={_}{\_}
        },
    columns/sub_domain/.style={
            string type, 
            column type= V{0.737500 in}, 
            string replace*={_}{\_}
        },
    columns/abbreviation/.style={
            string type, 
            column type= V{0.737500 in}, 
            string replace*={_}{\_}
        },
    columns/name/.style={
            string type, 
            column type= n{0.737500 in}, 
            string replace*={_}{\_}
        },
    columns/units/.style={
            string type, 
            column type= V{0.737500 in}, 
            string replace*={_}{\_}
        },
    columns/description/.style={
            string type, 
            string replace*={_}{\_},
            column type = V{2.000000 in}
        }
    ]{/Users/dyb/GitHub/C3S_311a_CDM/tables/observed_variable.csv}
\end{landscape}

\begin{landscape}
\pgfplotstabletypeset[
    empty header,
    begin table=\begin{longtable},
    %every head row/.style={output empty row},
    every nth row={1}{before row=\hline},
    every first row/.append style={
        before row={%
            % Initial caption
            \caption{units}
            \label{tab:DataTableUnits}\\
            % Initial column headers
            \hline\hline             \multicolumn{1} { V{1.540000 in}} { \textbf{\seqsplit{value}}} & 
            \multicolumn{1} { V{1.540000 in}} { \textbf{\seqsplit{units}}} & 
            \multicolumn{1} { V{1.540000 in}} { \textbf{\seqsplit{conventional\_abbreviation}}} & 
            \multicolumn{1} { V{1.540000 in}} { \textbf{\seqsplit{abbreviation\_in\_ASCII}}} & 
            \multicolumn{1} { V{1.540000 in}} { \textbf{\seqsplit{abbreviation\_in\_ITA2}}} & 
            \multicolumn{1} { V{3.000000 in} } {\textbf{\seqsplit{definition\_in\_base\_units}}} \\ \hline\hline \endfirsthead
            \multicolumn{6}{c}{Table \thetable\ units (cont.)} \\
            % column headers on additional pages
            \hline\hline             \multicolumn{1} {V{1.540000 in} } { \textbf{\seqsplit{value}}} & 
            \multicolumn{1} {V{1.540000 in} } { \textbf{\seqsplit{units}}} & 
            \multicolumn{1} {V{1.540000 in} } { \textbf{\seqsplit{conventional\_abbreviation}}} & 
            \multicolumn{1} {V{1.540000 in} } { \textbf{\seqsplit{abbreviation\_in\_ASCII}}} & 
            \multicolumn{1} {V{1.540000 in} } { \textbf{\seqsplit{abbreviation\_in\_ITA2}}} & 
            \multicolumn{1} { V{3.000000 in} } {\textbf{\seqsplit{definition\_in\_base\_units}}} \\ \hline\hline \endhead
            % Footer on 1st to penultimate pages
            \multicolumn{6}{r}{{Continued on next page}} \\
            \endfoot
            % Footer on last page of table
            \hline
            \multicolumn{6}{r}{{End of table}} \\ 
            \endlastfoot
        }
    },
    %
    end table=\end{longtable},
    col sep=tab,
    string type,
    columns/value/.style={
            string type, 
            column type= V{1.540000 in}, 
            string replace*={_}{\_}
        },
    columns/units/.style={
            string type, 
            column type= V{1.540000 in}, 
            string replace*={_}{\_}
        },
    columns/conventional_abbreviation/.style={
            string type, 
            column type= V{1.540000 in}, 
            string replace*={_}{\_}
        },
    columns/abbreviation_in_ASCII/.style={
            string type, 
            column type= V{1.540000 in}, 
            string replace*={_}{\_}
        },
    columns/abbreviation_in_ITA2/.style={
            string type, 
            column type= V{1.540000 in}, 
            string replace*={_}{\_}
        },
    columns/definition_in_base_units/.style={
            string type, 
            string replace*={_}{\_},
            column type = V{3.000000 in}
        }
    ]{/Users/dyb/GitHub/C3S_311a_CDM/tables/units.csv}
\end{landscape}

%\include {observation_code_tables}
\pgfplotstabletypeset[
    empty header,
    begin table=\begin{longtable},
    %every head row/.style={output empty row},
    every nth row={1}{before row=\hline},
    every first row/.append style={
        before row={%
            % Initial caption
            \caption{observation\_value\_significance (based on BUFR 0 08 023)}
            \label{tab:DataTableObservationvaluesignificance}\\
            % Initial column headers
            \hline\hline             \multicolumn{1} { V{2.800000 in}} { \textbf{\seqsplit{significance}}} & 
            \multicolumn{1} { V{4.000000 in} } {\textbf{\seqsplit{description}}} \\ \hline\hline \endfirsthead
            \multicolumn{2}{c}{Table \thetable\ observation\_value\_significance (cont.)} \\
            % column headers on additional pages
            \hline\hline             \multicolumn{1} {V{2.800000 in} } { \textbf{\seqsplit{significance}}} & 
            \multicolumn{1} { V{4.000000 in} } {\textbf{\seqsplit{description}}} \\ \hline\hline \endhead
            % Footer on 1st to penultimate pages
            \multicolumn{2}{r}{{Continued on next page}} \\
            \endfoot
            % Footer on last page of table
            \hline
            \multicolumn{2}{r}{{End of table}} \\ 
            \endlastfoot
        }
    },
    %
    end table=\end{longtable},
    col sep=tab,
    string type,
    columns/significance/.style={
            string type, 
            column type= V{2.800000 in}, 
            string replace*={_}{\_}
        },
    columns/description/.style={
            string type, 
            string replace*={_}{\_},
            column type = V{4.000000 in}
        }
    ]{/Users/dyb/GitHub/C3S_311a_CDM/tables/observation_value_significance.csv}

\input {spatial_representativeness}
%\pgfplotstabletypeset[
    empty header,
    begin table=\begin{longtable},
    %every head row/.style={output empty row},
    every nth row={1}{before row=\hline},
    every first row/.append style={
        before row={%
            % Initial caption
            \caption{z\_coordinate\_type}
            \label{tab:DataTableZcoordinatetype}\\
            % Initial column headers

%\documentclass{article}

\usepackage[a4paper, total={6in, 9.5in}]{geometry}
\usepackage{pgfplotstable} 
\usepackage{longtable}
\usepackage{booktabs}
\renewcommand{\familydefault}{\sfdefault}




\pgfplotstableset{
begin table=\begin{longtable},
end table=\end{longtable},
}


\title {Copernicus Climate Change Service - 311a Lot 2\\Defining a common data model}
\author {David I. Berry}
\date {23 June 2017}
\begin{document}
\maketitle
\section {Decode tables}

\begin{table}[h!]
  \begin{center}
    \caption{Z coordinate method}
    \pgfplotstabletypeset[
        col sep=tab, string type,
        header=true,
        every head row/.style={before row=\toprule\toprule, after row=\bottomrule\bottomrule\endhead}, 
        every last row/.style={after row=\bottomrule},
        after row=\hline,
    ]{/Users/dyb/Documents/Projects/C3S_311a_Lot2/WP2/CDM/github/tables/z_coordinate_method.csv}
  \end{center}
\end{table}

\end{document}

%\include {method_of_estimating_uncertainty}
%\include {sensor_configuration}
\input {automation_status}
\input {instrument_exposure_quality}
\documentclass{article}

\usepackage[a4paper, total={6in, 9.5in}]{geometry}
\usepackage{pgfplotstable} 
\usepackage{longtable}
\usepackage{booktabs}
\renewcommand{\familydefault}{\sfdefault}




\pgfplotstableset{
begin table=\begin{longtable},
end table=\end{longtable},
}


\title {Copernicus Climate Change Service - 311a Lot 2\\Defining a common data model}
\author {David I. Berry}
\date {23 June 2017}
\begin{document}
\maketitle
\section {Decode tables}

\begin{table}[h!]
  \begin{center}
    \caption{Conversion factor}
    \pgfplotstabletypeset[
        col sep=tab, string type,
        header=true,
        every head row/.style={before row=\toprule\toprule, after row=\bottomrule\bottomrule\endhead}, 
        every last row/.style={after row=\bottomrule},
        after row=\hline,
        columns={Value,description,Implementation},
        columns/Value/.style={string type, column type=l, string replace*={_}{}},
        columns/description/.style={string type, column type=l, string replace*={_}{}},
        columns/Implementation/.style={string type, column type=p{0.5\textwidth}, string replace*={_}{}}
    ]{/Users/dyb/Documents/Projects/C3S_311a_Lot2/WP2/CDM/github/tables/conversion_factor.csv}
  \end{center}
\end{table}

\end{document}

%\documentclass{article}

\usepackage[a4paper, total={6in, 9.5in}]{geometry}
\usepackage{pgfplotstable} 
\usepackage{longtable}
\usepackage{booktabs}
\renewcommand{\familydefault}{\sfdefault}




\pgfplotstableset{
begin table=\begin{longtable},
end table=\end{longtable},
}


\title {Copernicus Climate Change Service - 311a Lot 2\\Defining a common data model}
\author {David I. Berry}
\date {23 June 2017}
\begin{document}
\maketitle
\section {Decode tables}

\begin{table}[h!]
  \begin{center}
    \caption{Processing code}
    \pgfplotstabletypeset[
        col sep=tab, string type,
        header=true,
        every head row/.style={before row=\toprule\toprule, after row=\bottomrule\bottomrule\endhead}, 
        every last row/.style={after row=\bottomrule},
        after row=\hline,
        columns={Value,Description},
        columns/Value/.style={string type, column type=l, string replace*={_}{}},
        columns/Description/.style={string type, column type=p{0.5\textwidth}, string replace*={_}{}}
    ]{/Users/dyb/Documents/Projects/C3S_311a_Lot2/WP2/CDM/github/tables/processing_code.csv}
  \end{center}
\end{table}

\end{document}

\input {processing_level}
\begin{landscape}
\pgfplotstabletypeset[
    empty header,
    begin table=\begin{longtable},
    %every head row/.style={output empty row},
    every nth row={1}{before row=\hline},
    every first row/.append style={
        before row={%
            % Initial caption
            \caption{Adjustment}
            \label{tab:DataTable}\\
            % Initial column headers
            \hline\hline \multicolumn{1} { > {\centering}V{1.340000 in}} { \textbf{Value}} & 
\multicolumn{1} { > {\centering}V{1.340000 in}} { \textbf{Report ID}} & 
\multicolumn{1} { > {\centering}V{1.340000 in}} { \textbf{Observation ID}} & 
\multicolumn{1} { > {\centering}V{1.340000 in}} { \textbf{Adjustment}} & 
\multicolumn{1} { > {\centering}V{1.340000 in}} { \textbf{Reason}} & 
 \multicolumn{1} { > {\centering} V{3.000000 in} } {\textbf{Reference}} \\ \hline\hline \endfirsthead
            \multicolumn{6}{c}{Table \thetable\ Adjustment (cont.)} \\
            % column headers on additional pages
            \hline\hline \multicolumn{1} { > {\centering}V{1.340000 in} } { \textbf{Value}} & 
\multicolumn{1} { > {\centering}V{1.340000 in} } { \textbf{Report ID}} & 
\multicolumn{1} { > {\centering}V{1.340000 in} } { \textbf{Observation ID}} & 
\multicolumn{1} { > {\centering}V{1.340000 in} } { \textbf{Adjustment}} & 
\multicolumn{1} { > {\centering}V{1.340000 in} } { \textbf{Reason}} & 
 \multicolumn{1} { > {\centering} V{3.000000 in} } {\textbf{Reference}} \\ \hline\hline \endhead
            % Footer on 1st to penultimate pages
            \multicolumn{6}{r}{{Continued on next page}} \\
            \endfoot
            % Footer on last page of table
            \hline
            \multicolumn{6}{r}{{End of table}} \\ 
            \endlastfoot
        }
    },
    %
    end table=\end{longtable},
    col sep=tab,
    string type,
    columns/Value/.style={
            string type, 
            column type= V{1.340000 in}, 
            string replace*={_}{}
        },
    columns/Report ID/.style={
            string type, 
            column type= V{1.340000 in}, 
            string replace*={_}{}
        },
    columns/Observation ID/.style={
            string type, 
            column type= V{1.340000 in}, 
            string replace*={_}{}
        },
    columns/Adjustment/.style={
            string type, 
            column type= V{1.340000 in}, 
            string replace*={_}{}
        },
    columns/Reason/.style={
            string type, 
            column type= V{1.340000 in}, 
            string replace*={_}{}
        },
    columns/Reference/.style={
            string type, 
            string replace*={_}{},
            column type = V{3.000000 in}
        }
    ]{/Users/dyb/GitHub/C3S_311a_CDM/tables/adjustment.csv}
\end{landscape}

\input {traceability}
\begin{landscape}
\pgfplotstabletypeset[
    empty header,
    begin table=\begin{longtable},
    %every head row/.style={output empty row},
    every nth row={1}{before row=\hline},
    every first row/.append style={
        before row={%
            % Initial caption
            \caption{institute}
            \label{tab:DataTableInstitute}\\
            % Initial column headers
            \hline\hline             \multicolumn{1} { V{0.842857 in}} { \textbf{\seqsplit{institute}}} & 
            \multicolumn{1} { V{0.842857 in}} { \textbf{\seqsplit{name}}} & 
            \multicolumn{1} { V{0.842857 in}} { \textbf{\seqsplit{region}}} & 
            \multicolumn{1} { V{0.842857 in}} { \textbf{\seqsplit{sub\_region}}} & 
            \multicolumn{1} { V{0.842857 in}} { \textbf{\seqsplit{address}}} & 
            \multicolumn{1} { V{0.842857 in}} { \textbf{\seqsplit{contact}}} & 
            \multicolumn{1} { V{0.842857 in}} { \textbf{\seqsplit{contact\_email}}} & 
            \multicolumn{1} { V{3.000000 in} } {\textbf{\seqsplit{URL}}} \\ \hline\hline \endfirsthead
            \multicolumn{8}{c}{Table \thetable\ institute (cont.)} \\
            % column headers on additional pages
            \hline\hline             \multicolumn{1} {V{0.842857 in} } { \textbf{\seqsplit{institute}}} & 
            \multicolumn{1} {V{0.842857 in} } { \textbf{\seqsplit{name}}} & 
            \multicolumn{1} {V{0.842857 in} } { \textbf{\seqsplit{region}}} & 
            \multicolumn{1} {V{0.842857 in} } { \textbf{\seqsplit{sub\_region}}} & 
            \multicolumn{1} {V{0.842857 in} } { \textbf{\seqsplit{address}}} & 
            \multicolumn{1} {V{0.842857 in} } { \textbf{\seqsplit{contact}}} & 
            \multicolumn{1} {V{0.842857 in} } { \textbf{\seqsplit{contact\_email}}} & 
            \multicolumn{1} { V{3.000000 in} } {\textbf{\seqsplit{URL}}} \\ \hline\hline \endhead
            % Footer on 1st to penultimate pages
            \multicolumn{8}{r}{{Continued on next page}} \\
            \endfoot
            % Footer on last page of table
            \hline
            \multicolumn{8}{r}{{End of table}} \\ 
            \endlastfoot
        }
    },
    %
    end table=\end{longtable},
    col sep=tab,
    string type,
    columns/institute/.style={
            string type, 
            column type= V{0.842857 in}, 
            string replace*={_}{\_}
        },
    columns/name/.style={
            string type, 
            column type= V{0.842857 in}, 
            string replace*={_}{\_}
        },
    columns/region/.style={
            string type, 
            column type= V{0.842857 in}, 
            string replace*={_}{\_}
        },
    columns/sub_region/.style={
            string type, 
            column type= V{0.842857 in}, 
            string replace*={_}{\_}
        },
    columns/address/.style={
            string type, 
            column type= V{0.842857 in}, 
            string replace*={_}{\_}
        },
    columns/contact/.style={
            string type, 
            column type= V{0.842857 in}, 
            string replace*={_}{\_}
        },
    columns/contact_email/.style={
            string type, 
            column type= V{0.842857 in}, 
            string replace*={_}{\_}
        },
    columns/URL/.style={
            string type, 
            string replace*={_}{\_},
            column type = V{3.000000 in}
        }
    ]{/Users/dyb/GitHub/C3S_311a_CDM/tables/institute.csv}
\end{landscape}

\input {observing_frequency}
\documentclass{article}

\usepackage[a4paper, total={6in, 9.5in}]{geometry}
\usepackage{pgfplotstable} 
\usepackage{longtable}
\usepackage{booktabs}
\renewcommand{\familydefault}{\sfdefault}
\usepackage{showframe}



\pgfplotstableset{
  begin table=\begin{longtable},
  end table=\end{longtable},
}


\title {Copernicus Climate Change Service - 311a Lot 2\\Defining a common data model}
\author {David I. Berry}
\date {23 June 2017}
\begin{document}
\maketitle
\section {Decode tables}

\begin{table}[h!]
  \begin{center}
    \caption{Communication method}
    \pgfplotstabletypeset[
        col sep=tab,
        header=true,
        every head row/.style={
            before row=\toprule\toprule, after row=\bottomrule\bottomrule\endhead
        }, 
        every last row/.style={
            after row=\bottomrule
        },
        after row=\hline,
        columns={Value,Description},
        columns/Value/.style={
            string type, 
            column type=l, 
            string replace*={_}{}
        },
        columns/Description/.style={
            string type, 
            string replace*={_}{},
            column type = p{0.6\textwidth}
        }]{/Users/dyb/Documents/Projects/C3S_311a_Lot2/WP2/CDM/github/tables/communication_method.2.csv}
  \end{center}
\end{table}

\end{document}

%\pgfplotstabletypeset[
    empty header,
    begin table=\begin{longtable},
    %every head row/.style={output empty row},
    every nth row={1}{before row=\hline},
    every first row/.append style={
        before row={%
            % Initial caption
            \caption{measuring\_system\_model}
            \label{tab:DataTableMeasuringsystemmodel}\\
            % Initial column headers
            \hline\hline             \multicolumn{1} { V{1.900000 in}} { \textbf{\seqsplit{index}}} & 
            \multicolumn{1} { V{1.900000 in}} { \textbf{\seqsplit{measuring\_system\_model}}} & 
            \multicolumn{1} { V{3.000000 in} } {\textbf{\seqsplit{description}}} \\ \hline\hline \endfirsthead
            \multicolumn{3}{c}{Table \thetable\ measuring\_system\_model (cont.)} \\
            % column headers on additional pages
            \hline\hline             \multicolumn{1} {V{1.900000 in} } { \textbf{\seqsplit{index}}} & 
            \multicolumn{1} {V{1.900000 in} } { \textbf{\seqsplit{measuring\_system\_model}}} & 
            \multicolumn{1} { V{3.000000 in} } {\textbf{\seqsplit{description}}} \\ \hline\hline \endhead
            % Footer on 1st to penultimate pages
            \multicolumn{3}{r}{{Continued on next page}} \\
            \endfoot
            % Footer on last page of table
            \hline
            \multicolumn{3}{r}{{End of table}} \\ 
            \endlastfoot
        }
    },
    %
    end table=\end{longtable},
    col sep=tab,
    string type,
    columns/index/.style={
            string type, 
            column type= V{1.900000 in}, 
            string replace*={_}{\_}
        },
    columns/measuring_system_model/.style={
            string type, 
            column type= V{1.900000 in}, 
            string replace*={_}{\_}
        },
    columns/description/.style={
            string type, 
            string replace*={_}{\_},
            column type = V{3.000000 in}
        }
    ]{/Users/dyb/GitHub/C3S_311a_CDM/tables/measuring_system_model.csv}

\input {metadata_source}
%\begin{landscape}
\pgfplotstabletypeset[
    empty header,
    begin table=\begin{longtable},
    %every head row/.style={output empty row},
    every nth row={1}{before row=\hline},
    every first row/.append style={
        before row={%
            % Initial caption
            \caption{station\_configuration\_fields}
            \label{tab:DataTableStationconfigurationfields}\\
            % Initial column headers
            \hline\hline \multicolumn{1} { > {\centering}V{1.116667 in}} { \textbf{Value}} & 
\multicolumn{1} { > {\centering}V{1.116667 in}} { \textbf{Field}} & 
\multicolumn{1} { > {\centering}V{1.116667 in}} { \textbf{FieldName}} & 
\multicolumn{1} { > {\centering}V{1.116667 in}} { \textbf{Kind}} & 
\multicolumn{1} { > {\centering}V{1.116667 in}} { \textbf{Code Value}} & 
\multicolumn{1} { > {\centering}V{1.116667 in}} { \textbf{Abbreviation}} & 
 \multicolumn{1} { > {\centering} V{3.000000 in} } {\textbf{Description}} \\ \hline\hline \endfirsthead
            \multicolumn{7}{c}{Table \thetable\ station\_configuration\_fields (cont.)} \\
            % column headers on additional pages
            \hline\hline \multicolumn{1} { > {\centering}V{1.116667 in} } { \textbf{Value}} & 
\multicolumn{1} { > {\centering}V{1.116667 in} } { \textbf{Field}} & 
\multicolumn{1} { > {\centering}V{1.116667 in} } { \textbf{FieldName}} & 
\multicolumn{1} { > {\centering}V{1.116667 in} } { \textbf{Kind}} & 
\multicolumn{1} { > {\centering}V{1.116667 in} } { \textbf{Code Value}} & 
\multicolumn{1} { > {\centering}V{1.116667 in} } { \textbf{Abbreviation}} & 
 \multicolumn{1} { > {\centering} V{3.000000 in} } {\textbf{Description}} \\ \hline\hline \endhead
            % Footer on 1st to penultimate pages
            \multicolumn{7}{r}{{Continued on next page}} \\
            \endfoot
            % Footer on last page of table
            \hline
            \multicolumn{7}{r}{{End of table}} \\ 
            \endlastfoot
        }
    },
    %
    end table=\end{longtable},
    col sep=tab,
    string type,
    columns/Value/.style={
            string type, 
            column type= V{1.116667 in}, 
            string replace*={_}{\_}
        },
    columns/Field/.style={
            string type, 
            column type= V{1.116667 in}, 
            string replace*={_}{\_}
        },
    columns/FieldName/.style={
            string type, 
            column type= V{1.116667 in}, 
            string replace*={_}{\_}
        },
    columns/Kind/.style={
            string type, 
            column type= V{1.116667 in}, 
            string replace*={_}{\_}
        },
    columns/Code Value/.style={
            string type, 
            column type= V{1.116667 in}, 
            string replace*={_}{\_}
        },
    columns/Abbreviation/.style={
            string type, 
            column type= V{1.116667 in}, 
            string replace*={_}{\_}
        },
    columns/Description/.style={
            string type, 
            string replace*={_}{\_},
            column type = V{3.000000 in}
        }
    ]{/Users/dyb/GitHub/C3S_311a_CDM/tables/station_configuration_fields.csv}
\end{landscape}

%\begin{landscape}
\pgfplotstabletypeset[
    empty header,
    begin table=\begin{longtable},
    %every head row/.style={output empty row},
    every nth row={1}{before row=\hline},
    every first row/.append style={
        before row={%
            % Initial caption
            \caption{Profile configuration fields}
            \label{tab:DataTable}\\
            % Initial column headers
            \hline\hline \multicolumn{1} { > {\centering}V{0.837500 in}} { \textbf{Value}} & 
\multicolumn{1} { > {\centering}V{0.837500 in}} { \textbf{FieldNumber}} & 
\multicolumn{1} { > {\centering}V{0.837500 in}} { \textbf{FieldName}} & 
\multicolumn{1} { > {\centering}V{0.837500 in}} { \textbf{Type}} & 
\multicolumn{1} { > {\centering}V{0.837500 in}} { \textbf{Code Value}} & 
\multicolumn{1} { > {\centering}V{0.837500 in}} { \textbf{Abbreviation}} & 
\multicolumn{1} { > {\centering}V{0.837500 in}} { \textbf{Description}} & 
\multicolumn{1} { > {\centering}V{0.837500 in}} { \textbf{StartDate}} & 
 \multicolumn{1} { > {\centering} V{3.000000 in} } {\textbf{EndDate}} \\ \hline\hline \endfirsthead
            \multicolumn{9}{c}{Table \thetable\ Profile configuration fields (cont.)} \\
            % column headers on additional pages
            \hline\hline \multicolumn{1} { > {\centering}V{0.837500 in} } { \textbf{Value}} & 
\multicolumn{1} { > {\centering}V{0.837500 in} } { \textbf{FieldNumber}} & 
\multicolumn{1} { > {\centering}V{0.837500 in} } { \textbf{FieldName}} & 
\multicolumn{1} { > {\centering}V{0.837500 in} } { \textbf{Type}} & 
\multicolumn{1} { > {\centering}V{0.837500 in} } { \textbf{Code Value}} & 
\multicolumn{1} { > {\centering}V{0.837500 in} } { \textbf{Abbreviation}} & 
\multicolumn{1} { > {\centering}V{0.837500 in} } { \textbf{Description}} & 
\multicolumn{1} { > {\centering}V{0.837500 in} } { \textbf{StartDate}} & 
 \multicolumn{1} { > {\centering} V{3.000000 in} } {\textbf{EndDate}} \\ \hline\hline \endhead
            % Footer on 1st to penultimate pages
            \multicolumn{9}{r}{{Continued on next page}} \\
            \endfoot
            % Footer on last page of table
            \hline
            \multicolumn{9}{r}{{End of table}} \\ 
            \endlastfoot
        }
    },
    %
    end table=\end{longtable},
    col sep=tab,
    string type,
    columns/Value/.style={
            string type, 
            column type= V{0.837500 in}, 
            string replace*={_}{}
        },
    columns/FieldNumber/.style={
            string type, 
            column type= V{0.837500 in}, 
            string replace*={_}{}
        },
    columns/FieldName/.style={
            string type, 
            column type= V{0.837500 in}, 
            string replace*={_}{}
        },
    columns/Type/.style={
            string type, 
            column type= V{0.837500 in}, 
            string replace*={_}{}
        },
    columns/Code Value/.style={
            string type, 
            column type= V{0.837500 in}, 
            string replace*={_}{}
        },
    columns/Abbreviation/.style={
            string type, 
            column type= V{0.837500 in}, 
            string replace*={_}{}
        },
    columns/Description/.style={
            string type, 
            column type= V{0.837500 in}, 
            string replace*={_}{}
        },
    columns/StartDate/.style={
            string type, 
            column type= V{0.837500 in}, 
            string replace*={_}{}
        },
    columns/EndDate/.style={
            string type, 
            string replace*={_}{},
            column type = V{3.000000 in}
        }
    ]{/Users/dyb/Documents/Projects/C3S_311a_Lot2/WP2/CDM/github/tables/profile_configuration_fields.csv}
\end{landscape}

%\pgfplotstabletypeset[
    empty header,
    begin table=\begin{longtable},
    %every head row/.style={output empty row},
    every nth row={1}{before row=\hline},
    every first row/.append style={
        before row={%
            % Initial caption
            \caption{product\_level (NA)}
            \label{tab:DataTableProductlevel}\\
            % Initial column headers
            \hline\hline             \multicolumn{1} { V{0.933333 in}} { \textbf{\seqsplit{element\_name}}} & 
            \multicolumn{1} { V{0.933333 in}} { \textbf{\seqsplit{kind}}} & 
            \multicolumn{1} { V{0.933333 in}} { \textbf{\seqsplit{external\_table}}} & 
            \multicolumn{1} { V{4.000000 in} } {\textbf{\seqsplit{description}}} \\ \hline\hline \endfirsthead
            \multicolumn{4}{c}{Table \thetable\ product\_level (cont.)} \\
            % column headers on additional pages
            \hline\hline             \multicolumn{1} {V{0.933333 in} } { \textbf{\seqsplit{element\_name}}} & 
            \multicolumn{1} {V{0.933333 in} } { \textbf{\seqsplit{kind}}} & 
            \multicolumn{1} {V{0.933333 in} } { \textbf{\seqsplit{external\_table}}} & 
            \multicolumn{1} { V{4.000000 in} } {\textbf{\seqsplit{description}}} \\ \hline\hline \endhead
            % Footer on 1st to penultimate pages
            \multicolumn{4}{r}{{Continued on next page}} \\
            \endfoot
            % Footer on last page of table
            \hline
            \multicolumn{4}{r}{{End of table}} \\ 
            \endlastfoot
        }
    },
    %
    end table=\end{longtable},
    col sep=tab,
    string type,
    columns/element_name/.style={
            string type, 
            column type= n{0.933333 in}, 
            string replace*={_}{\_}
        },
    columns/kind/.style={
            string type, 
            column type= V{0.933333 in}, 
            string replace*={_}{\_}
        },
    columns/external_table/.style={
            string type, 
            column type= n{0.933333 in}, 
            string replace*={_}{\_}
        },
    columns/description/.style={
            string type, 
            string replace*={_}{\_},
            column type = V{4.000000 in}
        }
    ]{/Users/dyb/GitHub/C3S_311a_CDM/tables/product_level.csv}

%\include {product_status}
\input {source_format}
%\pgfplotstabletypeset[
    empty header,
    begin table=\begin{longtable},
    %every head row/.style={output empty row},
    every nth row={1}{before row=\hline},
    every first row/.append style={
        before row={%
            % Initial caption
            \caption{source\_configuration\_fields}
            \label{tab:DataTableSourceconfigurationfields}\\
            % Initial column headers
            \hline\hline             \multicolumn{1} { V{1.266667 in}} { \textbf{\seqsplit{field\_id}}} & 
            \multicolumn{1} { V{1.266667 in}} { \textbf{\seqsplit{field\_name}}} & 
            \multicolumn{1} { V{1.266667 in}} { \textbf{\seqsplit{kind}}} & 
            \multicolumn{1} { V{3.000000 in} } {\textbf{\seqsplit{description}}} \\ \hline\hline \endfirsthead
            \multicolumn{4}{c}{Table \thetable\ source\_configuration\_fields (cont.)} \\
            % column headers on additional pages
            \hline\hline             \multicolumn{1} {V{1.266667 in} } { \textbf{\seqsplit{field\_id}}} & 
            \multicolumn{1} {V{1.266667 in} } { \textbf{\seqsplit{field\_name}}} & 
            \multicolumn{1} {V{1.266667 in} } { \textbf{\seqsplit{kind}}} & 
            \multicolumn{1} { V{3.000000 in} } {\textbf{\seqsplit{description}}} \\ \hline\hline \endhead
            % Footer on 1st to penultimate pages
            \multicolumn{4}{r}{{Continued on next page}} \\
            \endfoot
            % Footer on last page of table
            \hline
            \multicolumn{4}{r}{{End of table}} \\ 
            \endlastfoot
        }
    },
    %
    end table=\end{longtable},
    col sep=tab,
    string type,
    columns/field_id/.style={
            string type, 
            column type= V{1.266667 in}, 
            string replace*={_}{\_}
        },
    columns/field_name/.style={
            string type, 
            column type= V{1.266667 in}, 
            string replace*={_}{\_}
        },
    columns/kind/.style={
            string type, 
            column type= V{1.266667 in}, 
            string replace*={_}{\_}
        },
    columns/description/.style={
            string type, 
            string replace*={_}{\_},
            column type = V{3.000000 in}
        }
    ]{/Users/dyb/GitHub/C3S_311a_CDM/tables/source_configuration_fields.csv}

%\include {manufacturer}
%\include {sensor_type}
%\include {sensor_model}
\input {observing_method}
\input {sampling_strategy}
\input {calibration_status}
%\begin{landscape}
\pgfplotstabletypeset[
    empty header,
    begin table=\begin{longtable},
    %every head row/.style={output empty row},
    every nth row={1}{before row=\hline},
    every first row/.append style={
        before row={%
            % Initial caption
            \caption{sensor\_configuration\_fields}
            \label{tab:DataTableSensorconfigurationfields}\\
            % Initial column headers
            \hline\hline             \multicolumn{1} { V{1.283333 in}} { \textbf{\seqsplit{value}}} & 
            \multicolumn{1} { V{1.283333 in}} { \textbf{\seqsplit{field}}} & 
            \multicolumn{1} { V{1.283333 in}} { \textbf{\seqsplit{parameter}}} & 
            \multicolumn{1} { V{1.283333 in}} { \textbf{\seqsplit{field\_name}}} & 
            \multicolumn{1} { V{1.283333 in}} { \textbf{\seqsplit{type}}} & 
            \multicolumn{1} { V{1.283333 in}} { \textbf{\seqsplit{code\_value}}} & 
            \multicolumn{1} { V{3.000000 in} } {\textbf{\seqsplit{description}}} \\ \hline\hline \endfirsthead
            \multicolumn{7}{c}{Table \thetable\ sensor\_configuration\_fields (cont.)} \\
            % column headers on additional pages
            \hline\hline             \multicolumn{1} {V{1.283333 in} } { \textbf{\seqsplit{value}}} & 
            \multicolumn{1} {V{1.283333 in} } { \textbf{\seqsplit{field}}} & 
            \multicolumn{1} {V{1.283333 in} } { \textbf{\seqsplit{parameter}}} & 
            \multicolumn{1} {V{1.283333 in} } { \textbf{\seqsplit{field\_name}}} & 
            \multicolumn{1} {V{1.283333 in} } { \textbf{\seqsplit{type}}} & 
            \multicolumn{1} {V{1.283333 in} } { \textbf{\seqsplit{code\_value}}} & 
            \multicolumn{1} { V{3.000000 in} } {\textbf{\seqsplit{description}}} \\ \hline\hline \endhead
            % Footer on 1st to penultimate pages
            \multicolumn{7}{r}{{Continued on next page}} \\
            \endfoot
            % Footer on last page of table
            \hline
            \multicolumn{7}{r}{{End of table}} \\ 
            \endlastfoot
        }
    },
    %
    end table=\end{longtable},
    col sep=tab,
    string type,
    columns/value/.style={
            string type, 
            column type= V{1.283333 in}, 
            string replace*={_}{\_}
        },
    columns/field/.style={
            string type, 
            column type= V{1.283333 in}, 
            string replace*={_}{\_}
        },
    columns/parameter/.style={
            string type, 
            column type= V{1.283333 in}, 
            string replace*={_}{\_}
        },
    columns/field_name/.style={
            string type, 
            column type= n{1.283333 in}, 
            string replace*={_}{\_}
        },
    columns/type/.style={
            string type, 
            column type= V{1.283333 in}, 
            string replace*={_}{\_}
        },
    columns/code_value/.style={
            string type, 
            column type= V{1.283333 in}, 
            string replace*={_}{\_}
        },
    columns/description/.style={
            string type, 
            string replace*={_}{\_},
            column type = V{3.000000 in}
        }
    ]{/Users/dyb/GitHub/C3S_311a_CDM/tables/sensor_configuration_fields.csv}
\end{landscape}






\end{document}

